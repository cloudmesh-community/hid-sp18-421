% status: 100
% chapter: Hadoop Technology

\title{HBase}


\author{Priyadarshini Vijjigiri}
\affiliation{%
  \institution{Indiana University}
  \streetaddress{Smith Research Center}
  \city{Bloomington} 
  \state{IN} 
  \postcode{47404}
}
\email{trovato@corporation.com}


% The default list of authors is too long for headers}
\renewcommand{\shortauthors}{P. Vijjigiri}


\begin{abstract}

HBase is 'NoSql' database designed after Googles BigTable to handle large datasets
through the use of coloumn orientation. In this paper we examine HBase by 
understanding the architecture, key features, components, data flow and limitations.

\end{abstract}

\keywords{hid-sp18-421, Column oriented database}

\maketitle

\section{Introduction}

HBase is a Nosql data base with key-value store. It is developed as a part of
Apache Hadoop Project. It is a column oriented Database Management systems which
is built on top of Hadoop Distributed file system. Googles Bigtable is a googles
NoSql database which is built to handle and quick random acess large data bases
with billions of rows and millions of columns. HBase is similar to Bigtable and
built on HDFS. HBase is an open source, distributed, versioned  and non-
relational database after bigtable~\cite {hid-sp18-421-HBase-intro}. Unlike
Nosql databases it provides consistency in read and write which makes it
different from even Nosql databases.

 HBase has many features~\cite{hid-sp18-421-HBase-feature}
 \begin{description}
  \item [1.] Scalability is the capability of application to add more machines without 
 actually changing the source code. HBase is linearly and also modularly 
 scalable. It can add more machines without changing the application code.

 \item [2.] As stated already HBase is consistent in reading and writing among the CURD 
 operations.

 \item [3.] It has a capability of providing automatic and configurable sharding tables.

 \item [4.] HBase provides automatic failure support from Region Servers.

 \item [5.] HBase has easy Java API for client.
 \end{description}


\section{Column Oriented database}

Column oriented database has different definition compared to row oriented one. 
It uses columns to store data tables that is each field of all the records in 
the table is stored  seperately where as row oriented database stores each 
record. Saving fields rather than records will have efficiency in computations.
column oriented database makes it easy access of required data among billions of 
data. 


 \section{Data Storage}

 HBase stores the data in key-value pair. Since it is column oriented database 
 for every data that is entered into HBase needs a row key. A row is broken into
 groupings called column families. These column families group similar or 
 frequently accessed data together. A row key uniquely identifies a row's data.

\section{History}

 After Google released paper on BigTable in 2006, Hadoop contributed a prototype
of HBase in Feb 2007 and first version of HBase is released in the same year 
october. HBase became a noticeable project in Jan 2008 and released subsequent 
versions in two years. In 2010 HBase became a top level project of 
Apache~\cite{hid-sp18-421-HBase-history}.

\section{Architecture}

HBase works on the concept of Master and Slave architecture. It has 3
components they are the HMaster, Region Servers and ZooKeeper.   HMaster
provides administrative functionalities like creation and deletion of tables
like handling the data by splitting and Region servers act as slaves. Region
Server communicates with HMaster for the actual files. Regionservers will have a
memstore where the cache of the file but actual HFile  would be there in Hadoop
Distributed File systems. And the log files get logged  into WAL of HDFS, so
whenever any of the Regionserver goes down we can rebuild  everything using the
log file. Zookeeper performs distribution coordination.

 Data in HBase table is divided horizontally into different Regions. These 
regions are sorted across different region service. one region server serves 
thousands of regions. The size of the regionservers are 1GB by default which 
can be controlled as per the requirement. Every region is sorted according to 
the key and the content of the region specified by start and the end key. HBase
is a key value store where value is a set of column families with a key being 
unique identifier of the records. The client can read from region server any 
quantity of data and any key they are looking for. The Regions are the horizontal
subsets of the table which are stored in Region server and are sorted according 
to the keys. 

 Region Servers and HMaster sends information to the zookeeper regularly. 
Zookeeper maintains which among them are active and inactive. So multiple 
HMasters help to rescue during down time which is detected my zookeeper. 

 HMaster is responsible for create and delete tables which are administrative 
 tasks and the client will connect to HMaster.

 \section{Meta Table} Meta table contains the information that in which region or
 on which region the start key and end key will be found. If we donot have the 
 table it will scan every region of the particular table to get a data. when we 
 wanted to get some data for a particular key and if you have the meta table then 
 you can directly go to the specified regionserver.
 \begin{description}
 \item GET is the operation which client performs to read a particular entry from
  HBase table.
\item PUT is the operation performed by the client to add some data to the HBase
 table.
\end{description}
 when ever these operations are required the client connects to the zookeeper to
 get the address of the meta table, on getting the address of the meta table it 
 stores the details of the data present in the meta table and looks upto that 
 particular data. 



\section{Key Features}

\subsection{Multidimensionality}

Traditional RDMS is two dimensional that means we need row and column to access 
data where as accessing HBase requires row key, column family name and the column
name. Also it is a sorted Map, Sparse and consistent. 

\subsection{Bigdata}

Architecture of HBase is completely distributed and so HBase can work on very 
large scale data with high speed. 

\subsection{Scalability} 

Facebook Messenger platform collaborates all the messaging platforms together 
like email, chat, SMS into real-time conversation and so it wanted robust and 
scalable infrastructure to facilitate these services. It has to store 
datasets so it requires dataset which has high processing and high performance
so that it can serve millions of requests at a time. Previously Facebook used 
Cassandra but after many discussions it shifted from Cassandra to HBase because
of its good scalability and simpler consistency model. 


\subsection{Table Operations} 

Zookeeper service helps to forward clients read request directly to RegionServer.
``An HBase write to a single row is atomic, meaning the whole operation either 
succeeds or fails, even if the write occurs across column families. A write 
operation to multiple rows however is not atomic, some row writes may succeed 
while others fail.''~\cite{hid-sp18-421-HBase-tableoper}. Multiple values can be
stored in HBase with timestamp and associated version, live values can be 
specified for cells which tells HBase to delete cells at intervals. Deletion 
request is written as tombstone marker to the blocks of data storing the row and 
the marked data is removed in the next major compaction.

\subsection{Compactions} 

\subsubsection{Minor Compactions}

Data is first written to memstore in HBase after it is full data is written to 
store-file on disk and marked read-only and when it reaches its threshold a minor
compaction occurs that merges multiple storefiles.

\subsubsection{Major Compactions}
Major Compaction runs for every 24 hours which merges all the storefiles into 
single storefile. Also any row marked as tombstone are deleted
~\cite{hid-sp18-421-HBase-tableoper}.


\subsection{Security and Access Management} 

HBase provides multiple mechanisms to provide high security and to easy control
or monitor data access to user.

\section{Limitations and Problems with HBase}
\begin{description}
  \item Only one HMaster is active at a time and since it acts as cluster for 5000 plus 
nodes if it goes down inorder to activate other HMaster it will take time.
 \item Cross data or joint operations cannot be done using HBase.
 \item Large data is available in RDBMS database source so while migrating data
 from such sources it may take longer time to create new design.
 \item Since HBase is not Sql we have to integrate some sql layers for querying.
 \item Real time queries and sorting is limited by storage memory.
 \item HBase has expensive hardware and for its performance it requires high 
 memory machines~\cite{hid-sp18-421-HBase-limitations}.
\end{description}

\section{Read and Write Path} 

\subsection{Write} Write or Update operation updates the in-memory data
(Memstore) and a local disk file(WAL -Write Ahead Log) which is used if incase
the memory crashes. Over the period of time, the data in Memstore is flushed to
the HDFS and the Memstore and WAL are cleared.

\subsection{Read} Read operation first looks up the Memstore first and if it is
not found then it looks in Mem cache(RAM part to keep frequently accessed data).
If it is not found in Memstore and Mem cache then it looks into HDFS.

\section{When HBase can be used?} HBase can be used when we have High volume
data, Unstructured, when we need Near Real time Operations, when we have to
perform Column oriented operations, and when high scalability is required.


\section{Conclusion} HBase is open source, column oriented database which
handles large datasets with billions of rows. Architecture allows it random and
fast read and write operations which can be scaled to even million/sec. HBase is
poor in cross and join operations.

\begin{acks}

The author would like to thank Dr.~Gregor~von~Laszewski for his support and
suggestions to write this paper and improve its content.

\end{acks}


\bibliographystyle{ACM-Reference-Format}
\bibliography{report} 


